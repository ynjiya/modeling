\usepackage{cmap}
\usepackage[T2A]{fontenc}
\usepackage[utf8]{inputenc}
\usepackage[english,russian]{babel}

\usepackage{amsmath}
\usepackage{amsfonts}

\usepackage{geometry}
\geometry{left=30mm}
\geometry{right=15mm}
\geometry{top=20mm}
\geometry{bottom=20mm}

\usepackage{titlesec}
\titleformat{\section}
{\normalsize\bfseries}
{\thesection}
{1em}{}
\titlespacing*{\chapter}{0pt}{10pt}{5pt}
\titlespacing*{\section}{\parindent}{*4}{*4}
\titlespacing*{\subsection}{\parindent}{*4}{*4}

\usepackage{setspace}
\onehalfspacing

\frenchspacing
\usepackage{indentfirst}

\usepackage{titlesec}
\titleformat{\chapter}{\large\bfseries}{\thechapter}{5pt}{\large\bfseries}
\titleformat{\section}{\bfseries}{\thesection}{5pt}{\bfseries}

\usepackage{xcolor}


\usepackage{pgfplots}
\usetikzlibrary{datavisualization}
\usetikzlibrary{datavisualization.formats.functions}
\usepackage{listings}
\usepackage{xcolor}

\definecolor{forestgreen}{rgb}{0.10, 0.45, 0.10}

\lstdefinestyle{mstyle}{
    language=Python,
    backgroundcolor=\color{white},
    basicstyle=\footnotesize\ttfamily,
    keywordstyle=\color{blue},
    stringstyle=\color{red},
    commentstyle=\color{gray},
    numbers=left,
    numberstyle=\tiny,
    stepnumber=1,
    numbersep=5pt,
    frame=single,
    tabsize=4,
    captionpos=t,
    breaklines=true,
    breakatwhitespace=true,
    columns=fullflexible,
    texcl=true
}


\usepackage{graphicx}
\newcommand{\img}[3] {
    \begin{figure}[h]
        \centering
        \includegraphics[height=#1]{#2}
        \caption{#3}
        \label{img:#2}
    \end{figure}
}

\newcommand{\imgw}[3] {
    \begin{figure}[h]
        \centering
        \includegraphics[width=#1]{#2}
        \caption{#3}
        \label{img:#2}
    \end{figure}
}

\usepackage[justification=centering]{caption}
\usepackage[unicode,pdftex]{hyperref}
\hypersetup{hidelinks}
\newcommand{\code}[1]{\texttt{#1}}
\usepackage{icomma}
\usepackage{csvsimple}
\usepackage{svg}

\newcommand\Tstrut{\rule{0pt}{2.6ex}}       % "top" strut
\newcommand\Bstrut{\rule[-0.9ex]{0pt}{0pt}} % "bottom" strut
\newcommand{\TBstrut}{\Tstrut\Bstrut} % top&bottom struts
