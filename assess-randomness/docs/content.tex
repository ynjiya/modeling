\textbf{Задание}

Реализовать собственный критерий оценки случайной последовательности. Сравнить результаты работы данного критерия на одноразрядных, двухразрядных и трехразрядных последовательностях целых чисел. Последовательности получать алгоритмическим и табличным способами.


\textbf{Критерий} \\
В качестве базовой гипотезы прининята идея о том, что каждое число в случайной последовательности имеет примерно равный шанс появления,
также 
разность между последующим и предыдущим членами должен измениться.

Результат критерия выражается в интервале $(-1 ; 1)$ (в выводе программы величина умножена на 100)

Проходя по членам последовательности находим частоту появления каждого уникального члена, и если разница между $x_{i+1}$ и $x_{i}$ равна разнице между $x_i$ и $x_{i-1}$ то уменьшаем значение оценки.

После этого вычисляется энтропия, что является мерой неоднородности (примеси) или неопределенности множества, вычисляется следующей формулой:
\begin{equation}
    E = - \sum_{i=1}^{K} p_i \, \log_2(p_i),
\end{equation}
где $p_i$ - число повторений $i$-того уникального члена делённый на длину последовательности, $K$ - количество уникальных член.
Хорошая случайная последовательность  имеет высокую примесь или неопределенность, соответсвенно, его энтропия близка к 1 -- максимально возможное значение энтропии. 
Энтропия последовательности, состоящего только с 1 уникального элемента, равна $E = -1 \cdot log_2(1) = 0$, что и является минимальным возможным значением энтропии. 

Таким образом, по каждому разряду член последовательности с малого до высшего (предполагается, что все элементы последовательности одноразрядные) вычисляется оценка.
В итоге вычисляется среднее значение от полученных оценок, что и является результатам критерия.

Оценка будет низкой:
\begin{itemize}
    \item если в последовательности мало уникальных член, то есть последовательность создана из малого набора чисел;
    \item встречается возрастающая/убывающая/неизменящаяся последовательность.

\end{itemize}


\chapter{Код реализации критерия}

\lstinputlisting[style=mstyle, caption=main.py -- Реализация критерия, linerange={28-66}]{../main.py}





\chapter{Результаты работы программы}

\imgw{170mm}{a-123}{Пример работы программы}

\imgw{170mm}{a--123}{Пример работы программы -- оценки одноразрядных последовательностей низкие, т.к. число уникальных член низкое}

\imgw{160mm}{012}{Пример работы программы --  последовательность возрастающая, но создана с большего набора чисел}


\imgw{160mm}{0122}{Пример работы программы --  последовательность возрастающая, но создана с малого набора чисел}

\imgw{160mm}{987}{Пример работы программы --  последовательность убывающая}

\imgw{160mm}{111}{Пример работы программы --  последовательность из одного числа}

\imgw{160mm}{1122}{Пример работы программы --  последовательность из 2 чисел}

\imgw{160mm}{1123}{Пример работы программы --  последовательность повторяющихся и возрастающих чисел}

\imgw{160mm}{1129}{Пример работы программы --  последовательность повторяющихся и рандомных чисел}

\imgw{160mm}{1919}{Пример работы программы}


\imgw{160mm}{7172}{Пример работы программы --  последовательность 2-х разрядных чисел (оценка низкая, т.к. первые разряды совпадают, что дает низкую оценку в последнем цикле $access\_randomnes()$))}

\imgw{160mm}{721732.png}{Пример работы программы --  последовательность 3-х разрядных чисел (оценка выше, т.к. 2 и 3 разряды не совпадают)}

\imgw{160mm}{321732.png}{Пример работы программы --  последовательность 3-х разрядных чисел (оценка выше, т.к. 2 и 3 разряды не совпадают, в последовательности из первых разрядов много уникальные числа)}


\imgw{170mm}{3-70}{Пример работы программы --  для 3-х разрядных чисел оценки невысокие, т.к. в последовательности из первых разрядов мало уникальных чисел}


\chapter{Код программы}

\lstinputlisting[style=mstyle, caption=markov.py -- Реализация методов]{../main.py}

